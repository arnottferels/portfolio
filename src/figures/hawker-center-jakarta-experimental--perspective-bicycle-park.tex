%
\begin{figure}[H]
	\centering
	\includesvg[width=\linewidth]{src/graphics/hawker-center-jakarta-experimental--perspective-bicycle-park.svg}
	\caption*{%
		Bicycle park
	}
	\label{
		fig:hawker-center-jakarta-experimental--perspective-bicycle-park
	}
\end{figure}
