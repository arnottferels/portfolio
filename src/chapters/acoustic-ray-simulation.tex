\setboolean{IsHalfPage}{false}%
\setboolean{IsHalfPageLeftCol}{false}%
\setboolean{IsHalfPageRightCol}{false}%
\def\ChapterTitle{%
	Acoustic Ray Simulation
}
\def\ChapterUrl{%
	https://arnottferels.github.io/work/acoustic-ray-simulation
}
\def\ChapterDescription{%
	Designing and Evaluating the KAI-MICE Auditorium Design
}
\def\ChapterDetailsLine{%
	Professional Work -- 2023 | Computation; Auditorium Design; Acoustic Design | Bandung, Indonesia
}
\def\ChapterDetailsTabular{%
	\begin{tabular}{@{}ll}
		\textbf{Type}          & National Professional Competition by IAI West Java (Indonesian Institute of Architect West Java)    \\
		\textbf{Contributions} & Research, Conceptual Design, Acoustic Modeling, Analysis and Simulation, Scripting \& Visualization \\
		\textbf{Software}      & Rhino, Grasshopper, Pachyderm Acoustic \& Twinmotion                                                \\
		\textbf{Collaborators} & Robby D. Juliardi, Ekky Maulidin, Ghina Z \& Zulafa Azmi                                            \\
		\textbf{URL}           & \textcolor{blue}{\footnotesize\texttt{\href{\ChapterUrl}{\ChapterUrl}}}                             \\
	\end{tabular}
}
\def\ChapterAbstract{%
	This study details the creation of a specialized auditorium model, inspired by Architect's Data by Neufert. Employing Grasshopper for algorithmic modeling, parameters were refined, emphasizing ray distribution simulation for acoustic analysis via Ray Pachyderm Acoustical Simulation. Data visualization, featuring a heat map, illustrates ray counts at each step for seats. In conclusion, this method offers insights into sound ray behavior in acoustics.
}
\def\ChapterFrontmatter{%
	\chapter*{\ChapterTitle}\addcontentsline{toc}{chapter}{\ChapterTitle}
	\ChapterSetTocAddData{\ChapterDetailsLine}
	\ChapterSetDetailsData{\ChapterDescription}{\ChapterDetailsLine}{\ChapterDetailsTabular}
	\RuleAbstract%
	\ChapterAbstract
}
\StartTwoColumnLayout
\ChapterFrontmatter
\section*{%
  Method
 }
\begin{figure}[H]
	\centering
	\includesvg[width=\linewidth]{src/graphics/acoustic-ray-simulation--method.svg}
	\label{
		fig:acoustic-ray-simulation--method
	}
\end{figure}

\vfill
\section*{%
  Distribution of Sound Particles (Audio)
 }
\begin{figure}[H]
	\centering
	\includesvg[width=\linewidth]{src/graphics/acoustic-ray-simulation--distribution-01.svg}
	\label{
		fig:acoustic-ray-simulation--distribution-01
	}
\end{figure}

\columnbreak%
\section*{%
  Distribution of Sound Particles (Audio) -- \textit{(cont.)}
 }
\begin{figure}[H]
	\centering
	\includesvg[width=\linewidth]{src/graphics/acoustic-ray-simulation--distribution-02.svg}
	\caption*{%
		\parbox{0.8\linewidth}{%
			\centering
			The Heatmap Diagram and Distribution of Audio Sound Particles to Ear Height (EH) Levels in the Acoustic Ray Simulation Process reveal Arrival Time Delay (ATD) ranging from 0~to~0.16281~seconds in 21~steps.
		}
	}
	\vspace*{\baselineskip}
	\label{
		fig:acoustic-ray-simulation--distribution-02
	}
\end{figure}

The diagram depicts the simulation in a 1530-seat auditorium, recording 61,208~Ray Curves (RC) for ATD. Red areas signal more than 10 sound reflections, while blue and yellow indicate 2 and 6 reflections, optimizing sound based on seat positions.
\vfill
\section*{%
  Acoustic Material Recommendation -- Sound Absorption for Optimization
 }
\begin{minipage}[t]{0.55\linewidth}
	\begin{table}[H]
	\centering
	\renewcommand{\tabularxcolumn}[1]{m{#1}}
	\renewcommand{\arraystretch}{1.25}
	\footnotesize
	\begin{tabularx}{\linewidth}
		{%
			@{}c
			p{0.75cm}
			p{0.75cm}
			p{0.75cm}
			*{9}{>{\scriptsize\centering\arraybackslash}X}
			c
			@{}
		}
		\toprule
		\multirow{2}{*}{\centering No.}                           &
		\multirow{2}{*}{\centering Element}                       &
		\multirow{2}{*}{\centering Material}                      &
		\multirow{2}{*}{\centering Finishing}                     &
		\multicolumn{9}{c}{Absorption coef. (\% energy absorbed)} &
		\multirow{2}{*}{\centering Page Ref.}
		\\
		\cmidrule(lr){5-13}
		                                                          &
		                                                          &
		                                                          &
		                                                          &
		{\rotatebox{90}{\scriptsize 62.5 Hz}}                     &
		{\rotatebox{90}{\scriptsize 125 Hz}}                      &
		{\rotatebox{90}{\scriptsize 250 Hz}}                      &
		{\rotatebox{90}{\scriptsize 500 Hz}}                      &
		{\rotatebox{90}{\scriptsize 1K Hz}}                       &
		{\rotatebox{90}{\scriptsize 2K Hz}}                       &
		{\rotatebox{90}{\scriptsize 4K Hz}}                       &
		{\rotatebox{90}{\scriptsize 8K Hz}}                       &
		{\rotatebox{90}{\scriptsize Flatten All}}
		                                                          &
		\\
		\midrule
		1                                                         & Wall       & Rockwool 75mm & Fabric & - & 0.3  & 0.69 & 0.94 & 1    & 1    & 1    & - & 0.82 & 10 \\
		2                                                         & Floor      & Carpet        & Fabric & - & 0.1  & 0.15 & 0.25 & 0.3  & 0.3  & 0.3  & - & 0.23 & 1  \\
		3                                                         & Furniture  & Chair         & Fabric & - & 0.33 & 0.44 & 0.45 & 0.45 & 0.45 & 0.45 & - & 0.42 & 2  \\
		4                                                         & Ceiling    & Woodwool 50mm & Fabric & - & 0.3  & 0.4  & 0.5  & 0.85 & 0.5  & 0.65 & - & 0.53 & 3  \\
		5                                                         & LED screen & -             & -      & - & -    & -    & -    & -    & -    & -    & - & -    & 5  \\
		\bottomrule
	\end{tabularx}
	\vspace*{0.5\baselineskip}%
	\caption*{%
		\parbox{0.8\linewidth}{%
			\centering
			Acoustic material recommendations for absorption in auditorium spaces. Source: Acoustic Projects Study (Acoustic Traffic LLC, 2023).
		}
	}
\end{table}

	The table outlines acoustic material recommendations for optimizing sound in the auditorium, considering material types, finishes, and absorption coefficients. Selection criteria include sound resonance, durability, and aesthetics. Follow these guidelines for enhanced sound quality and material durability in the auditorium.
\end{minipage}
\hfill
\begin{minipage}[t]{0.425\linewidth}
	\begin{figure}[H]
	\centering
	\includesvg[width=\linewidth]{src/graphics/acoustic-ray-simulation--interior.svg}
	\caption*{%
		Implementation of acoustic materials in the auditorium space.
	}
	\label{
		fig:acoustic-ray-simulation--interior
	}
\end{figure}

	Acoustic materials in the auditorium are optimized for the best sound absorption. From Rockwool on the walls to carpet on the floor, every element contributes to improving the room's sound quality.
\end{minipage}
\EndTwoColumnLayout
\newpage
